%%%%%%%%%%%%%%%%%%%%%%%%%%%%%%%%%%%%%%%%%%%%%%%%%%
% SIXTH PAGE

\setnewpagemargins

Now we head for an upper bound ``$\Theta(G) \leq \sigma^{-1}$" for the Shannon capacity
of any graph $G$ that has an especially ``nice" orthonormal representation.
For this let $T = \{v^{(1)},\ldots,v^{(m)}\}$ be an orthonormal representation 
of $G$ in $\mathbb{R}^{s}$, where $v^{(i)}$ corresponds to the vertex $v_i$. We assume in
adition that all the vectors $v^{(i)}$ have the same angle $(\neq 90^{\circ})$ with the
vector $u := {\frac{1}{m}}(v^{(1)}+\ldots+v^{(m)})$, or equivalently that the inner product 

\[
 \langle v^{(i)}, u\rangle  = \sigma _T
\]

has the same value $\sigma_T \neq 0$ for all $i$. Let us call this value $\sigma_T$ the \textit{constant}
of the representation $T$. For the Lov\'asz umbrella that represents $C_5$ the 
condition $\langle v^{(i)},u \rangle = \sigma_T$ certainly holds, for $u = \overset{\rightarrow}{OS}$.\\
Now we proceed in the following three steps.\\
$\mathbf{(A)}$ Consider a probability distribution $x = (x_1,\ldots, x_m)$ on $V$ and set

\[
\mu(x) := \quad |x_1v^{(1)}+\ldots+x_mv^{(m)}|^{2},
\]

and

\[
{\mu_{T}}(G) := \underset{x}{\inf}\mu(x).
\]

Let $U$ be a largest independent set in $G$ with $|U| = a$, and define
$x_U = (x_1,\ldots,x_m)$ with $x_i = {\frac{1}{\alpha}}$ if $v_i\in U$ and $x_i = 0$  otherwise. Since all 
vectors $v^{(i)}$  have unit length and $\langle v^{(i)}, v^{(j)} \rangle = 0$ for any two non-adjacent 
vertices, we infer

\[
    \mu_T(G)\leq\mu(x_U)=|\sum_{i=1}^{m}x_iv^{(i)}|^2= \sum_{i=1}^{m} {x_i}^2 = \alpha {\frac{1}{\alpha^{2}}} = {\frac{1}{\alpha}}.
\]

Thus we have $\mu_{T}(G) \leq \alpha^{-1}$ , and therefore

\[
    \alpha(G) \leq {\frac{1}{\mu_{T}(G)}}.
\]

$\mathbf{(B)}$ Next we compute ${\mu_T}(G)$. We need the Cauchy-Schwarz inequality

\[
\langle a,b \rangle^2 \leq |a|^2 |b|^2
\]

for vectors $a,b \in \mathbb{R}^s$. Applied to $a = x_{1}v^{(1)}+\ldots+x_{m}v^{(m)}$ and $b=u$,
the inequality yields

\begin{equation}
  \langle x_{1}v^{(1)}+\ldots+x_{m}v^{(m)},u \rangle ^2 \leq \mu(x) |u|^2. \label{five} \label{five}
\end{equation}

By our assumption that $\langle v^{(i)}, u \rangle = \sigma_T$ for all $i$, we have

\[
\langle x_{1}v^{(1)}+\ldots+x_{m}v^{(m)},u\rangle = (x_1 + \ldots + x_m)\sigma_T = \sigma_T
\]

for any distribution $x$. Thus, in particular, this has to hold for the uniform 
distribution $(\frac{1}{m}, \ldots, \frac{1}{m})$,which implies $|u|^2 = \sigma_T$.
Hence (\ref{five}) reduces to 

\begin{equation*}
\sigma_T^2 \leq \mu(x)\sigma_T \quad \text{or} \quad \mu_{T}(G) \geq \sigma_T.
\end{equation*}

%%7th page
\setnewpagemargins
On the other hand, for $x = (\frac{1}{m}, \ldots, \frac{1}{m})$  we obtain

\[
\mu_{T}(G) \leq \mu(x) = |{\frac{1}{m}}(v^{(1)}+\ldots+v^{(m)})|^2 = |u|^2 = \sigma_T,
\]

and so we have proved 

\begin{equation}
\mu_{T}(G) = \sigma_T. \label{six}
\end{equation}

In summary, we have established the inequality 

\begin{equation}
    \alpha(G) \leq {\frac{1}{\sigma_T}} \label{seven}  
\end{equation}

for any orthonormal respresentation $T$ with constant $\sigma_T$.\\
$\mathbf{(C)}$ To extend this inequality to $\Theta(G)$, we proceed as before. Consider 
again the product $G \times H$ of two graphs. Let $G$ and $H$ have orthonormal 
representations $R$ and $S$ in $\mathbb{R}^r$ and $\mathbb{R}^s$, respectively, with constants $\sigma_R$ and $\sigma_S$.
Let $v = (v_1,\ldots,v_r)$ be a vector in $R$ and $w = (w_1,\ldots,w_s)$ be 
a vector in $S$. To the vertex in $G \times H$ corresponding to the pair $(v, w)$ we 
associate the vector

\[
vw^T := ({v_1}{w_1},\ldots,{v_1}{w_s},{v_2}{w_1},\ldots,{v_2}{w_s},\ldots,{v_r}{w_1},\ldots,{v_r}{w_s}) \in \mathbb{R}^{rs}.
\]

It is immediately checked that $R \times S := \{vw^T : v \in R, w \in S\}$ is an
orthonormal representation of $G \times H$ with constant $\sigma_R \sigma_S$. Hence by (\ref{six}) 
we obtain 

\[
\mu_{R \times S}(G \times H) = \mu_{R}(G)\mu_{S}(H).
\]

For $G^n = G \times \ldots \times G$ and the representation T with constant $\sigma_T$ this
means 

\[
\mu_{T^n}(G^n) = \mu_{T}(G)^n = \sigma_{T}^n
\]

and by (\ref{seven}) we obtain


\[
\alpha(G^n) \leq \sigma_{T}^{-n},  \quad   \sqrt[n]{\alpha(G^n)} \leq \sigma_{T}^{-1}.
\]

Taking all things together we have thus completed Lov\'asz' argument: 


\begin{thm}\label{theorem}
whenever $T = \{v^{(1)}, \ldots, v^{(m)}\}$) is an orthonormal\\ 
representation of $G$ with constant $\sigma_T$, then

\begin{equation}
    \Theta(G) \leq {\frac{1}{\sigma_T}}. \label{eight}
\end{equation}

\end{thm}

Looking at the Lov\'asz umbrella, we have $u = (0, 0, h = {\frac{1}{\sqrt[4]{5}}})^T$ and hence
$\sigma = \langle v^{(i)}, u \rangle = h^2 = {\frac{1}{\sqrt{5}}}$, which yields $\Theta(C_5) \leq \sqrt{5}$. Thus Shannon's 
problem is solved.


%8th page
\setnewpagemargins

Let us carry our discussion a little further. We see from (\ref{eight}) that the larger $\sigma_T$
is for a representation of $G$, the better a bound for $\Theta(G)$ we will get. Here 
is a method that gives us an orthonormal representation for \textit{any} graph $G$. 
To $G = (V, E)$ we associate the adjacency matrix $A = (a_{ij})$, which is 
defined as follows: Let $V = \{v_1,\ldots,v_m\}$, then we set

\[
a_{ij} := \left\{ \begin{array}{rcl}
  1 & \mbox{if $v_i v_j \in E$}\\
  0 & \mbox{otherwise.}
  \end{array}\right.
\]

$A$ is a real symmetric matrix with $0$'s in the main diagonal.\\ 
Now we need two facts from linear algebra. First, as a symmetric matrix, 
$A$ has $m$ real eigenvalues $\lambda_1 \geq \lambda_2 \geq \ldots \geq \lambda_m$ \spaceskip(some of which may be equal),
and the sum of the eigenvalues equals the sum of the diagonal 
entries of $A$, that is, 0. Hence the smallest eigenvalue must be negative 
(except in the trivial case when $G$ has no edges). {Let $p = |\lambda_m| = -\lambda_m$} be 
the absolute value of the smallest eigenvalue, and consider the matrix

\[
M := 1 + {\frac{1}{p}}A,
\]

where $I$ denotes the $(m \times m)$-identity matrix. This $M$ has the eigenvalues
$1+{\frac{\lambda_1}{p}} \geq 1+{\frac{\lambda_2}{p}} \geq \ldots \geq 1+{\frac{\lambda_m}{p}} =0$. Now we quote the second result (the principal axis theorem of linear algebra):
If $M = (m_{ij})$ is a real symmetric matrix with all eigenvalues $\geq 0$,
then there are vectors $v^{(1)}, \ldots, v^{(m)} \in \mathbb{R}^s$
for $s = \text{rank}(M)$, such that 

\[
m_{ij} = \langle v^{(i)}, v^{(j)} \rangle  \quad  (1 \leq i,j \leq m).
\]

In particular, for $M = I + {\frac{1}{p}}A$ we obtain

\[
\langle v^{(i)}, v^{(i)} \rangle = m_{ii} = 1  \quad \text{for all }i
\]

and

\[
\langle v^{(i)}, v^{(j)} \rangle = {\frac{1}{p}}a_{ij} \quad \text{for }i \neq j.
\]

Since $a_{ij} = 0 $ whenever $ v_i v_j \notin E$, we see that the 
vectors $v^{(1)},\ldots,v^{(m)}$ form indeed an orthonormal representation of $G$.\\[5pt]
Let us, finally, apply this construction to the $m$-cycles $C_m$ for odd $m > 5$. 
Here one easily computes $p = |\lambda_{min}| = 2\cos{\frac{\pi}{m}}$ (see the box). Every 
row of the adjacency matrix contains two $1$'s, implying that every row of 
the matrix $M$ sums to $1 + {\frac{2}{p}}$. For the representation $\{ v^{(1)}, \ldots, v^{(m)}\}$ this
means

\[ 
    \langle v^{(i)},v^{(1)}+ \ldots + v^{(m)} \rangle=1+{\frac{2}{p}}=1+{\frac{1}{\cos{\frac{\pi}{m}}}}
\]

and hence 

\[
\langle v^{(i)},u \rangle = {\frac{1}{m}}(1+(\cos{\frac{\pi}{m}})^{-1})=\sigma
\]

for all $i$. We can therefore apply our main result (\ref{eight}) and conclude

\begin{equation}
    \Theta(C_m) \leq {\frac{m}{1+(\cos{\frac{\pi}{m}})^{-1}}} \quad\quad\quad (\text{for $m \geq 5$ odd}).\label{nine}    
\end{equation}


%9th page
\setnewpagemargins

Notice that because of $\cos{\frac{\pi}{m}} < 1$ the bound (\ref{nine}) is better than the bound 
$\Theta(C_m) \leq {\frac{m}{2}}$ we found before. Note further $\cos{\frac{\pi}{5}} = {\frac{\tau}{2}}$, where $\tau = {\frac{\sqrt{5}+1}{2}}$
is the golden section. Hence for $m = 5$ we again obtain

\[
\Theta(C_5) \leq {\frac{5}{1 + {\frac{4}{\sqrt{5}+1}}}}={\frac{5(\sqrt{5}+1)}{5+\sqrt{5}}} = \sqrt{5}.
\]

The orthonormal representation given by this construction is, of course, 
precisely the ``Lov\'asz umbrella."\\
And what about $C_7, C_9,$ and the other odd cycles? By considering $\alpha(C_m^2)$,
$\alpha(C_m^3)$ and other small powers the lower bound ${\frac{m-1}{2}} \leq \Theta(C_m)$ ) can cer-
tainly be increased, but for no odd $m \geq 7$  do the best known lower bounds
agree with the upper bound given in (\ref{eight}). So, twenty years after Lov\'asz' 
marvelous proof of $\Theta(C_5) = \sqrt{5}$, these problems remain open and are 
considered very difficult \text{---} but after all we had this situation before.\\

\begin{mdframed}[nobreak=true,backgroundcolor=gray!15]
\vspace{8pt}
{\Large\textbf{The eigenvalues of $C_m$}}
\vspace{5pt}

Look at the adjacency matrix $A$ of the cycle $C_m$. To find the eigenvalues 
(and eigenvectors) we use the $m$-th roots of unity. These are 
given by $1, \zeta, \zeta^2, \ldots, \zeta^{m-1}$ for $\zeta = e^{\frac{2\pi i}{m}}$ --- see the box on page 25.\\ 
Let $\lambda \quad=\quad\zeta ^k$ \ \ be\ \  any\ \  of\ \  these roots, then we claim that
$(1, \lambda, \lambda^2, \ldots, \lambda^{m-1})^T$ is an eigenvector of $A$ to the eigenvalue $\lambda + \lambda^{-1}$.
In fact, by the set-up of $A$ we find\\

\[
A \begin{pmatrix} 1 \\ \lambda \\ \lambda^2 \\ \vdots \\ \lambda^{m-1} \end{pmatrix}=
\begin{pmatrix} \lambda + \lambda^{m-1} \\ \lambda^2 + 1 \\ \lambda^3 + \lambda \\ \vdots \\ 1+\lambda^{m-2} \end{pmatrix} =
(\lambda + \lambda^{-1}) \begin{pmatrix} 1 \\ \lambda \\ \lambda^2 \\ \vdots \\ \lambda^{m-1} \end{pmatrix} 
\]

Since the vectors $(1, \lambda, \ldots, \lambda^{m-1})$ are independent (they form a so-
called Vandermonde matrix) we conclude that for odd $m$

\begin{equation*}
\begin{aligned}
    \zeta^k + \zeta^{-k}&\quad = [(\cos(2k\pi / m)) + i\sin(2k\pi / m)]\\
&\quad\quad\ \  + [\cos(2k\pi/m) - i\sin(2k\pi / m)]\\
&\quad =  2\cos(2k\pi / m) \quad\quad\  (0 \leq k \leq \text{\tiny{$\frac{m-1}{2}$}})
\end{aligned}
\end{equation*}\\

are all the eigenvalues of $A$. Now the cosine is a decreasing function,
and So

\[
    2\cos({\frac{(m-1)\pi}{m}}) = -2\cos{\frac{\pi}{m}}
\]

is the smallest eigenvalue of $A$. 

\vspace{8pt}
\end{mdframed}


%%%%%%%%%%%%%%%%%%%%%%%%%%%%%%%%%%%%%%%%%%%%%%%%%%
% TENTH PAGE

\setnewpagemargins

% REFERENCES
\bibliography{refs.bib}
